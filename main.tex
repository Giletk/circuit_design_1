\documentclass[a4paper,14pt,oneside,openany]{memoir}

%%% Задаем поля, отступы и межстрочный интервал %%%
%#myadd
%шрифт для кода
\usepackage{fontspec}
\setmonofont{DejaVu Sans}
\newfontfamily{\cyrillicfonttt}{DejaVu Sans}
%%%
\usepackage[left=30mm, right=15mm, top=20mm, bottom=20mm]{geometry} % Пакет geometry с аргументами для определения полей
\pagestyle{plain} % Убираем стандарные для данного класса верхние колонтитулы с заголовком текущей главы, оставляем только номер страницы снизу по центру
\parindent=1.25cm % Абзацный отступ 1.25 см, приблизительно равно пяти знакам, как по ГОСТ
\usepackage{indentfirst} % Добавляем отступ к первому абзацу
%\linespread{1.3} % Межстрочный интервал (наиболее близко к вордовскому полуторному) - тут вместо этого используется команда OnehalfSpacing*

%%% Задаем языковые параметры и шрифт %%%

\usepackage[english, russian]{babel}                % Настройки для русского языка как основного в тексте
\babelfont{rm}[Language=Default]{Times New Roman}                     % TMR в качестве базового roman-щрифта

%%% Задаем стиль заголовков и подзаголовков в тексте %%%

\setsecnumdepth{subsection} % Номера разделов считать до третьего уровня включительно, т.е. нумеруются только главы, секции, подсекции
\renewcommand*{\chapterheadstart}{} % Переопределяем команду, задающую отступ над заголовком, чтобы отступа не было
\renewcommand*{\printchaptername}{} % Переопределяем команду, печатающую слово "Глава", чтобы оно не печалось
%\renewcommand*{\printchapternum}{} % То же самое для номера главы - тут не надо, номер главы оставляем
\renewcommand*{\chapnumfont}{\normalfont\bfseries} % Меняем стиль шрифта для номера главы: нормальный размер, полужирный
\renewcommand*{\afterchapternum}{\hspace{1em}} % Меняем разделитель между номером главы и названием
\renewcommand*{\printchaptertitle}{\normalfont\bfseries\centering\MakeUppercase} % Меняем стиль написания для заголовка главы: нормальный размер, полужирный, центрированный, заглавными буквами
\setbeforesecskip{20pt} % Задаем отступ перед заголовком секции
\setaftersecskip{20pt} % Ставим такой же отступ после заголовка секции
\setsecheadstyle{\raggedright\normalfont\bfseries} % Меняем стиль написания для заголовка секции: выравнивание по правому краю без переносов, нормальный размер, полужирный
\setbeforesubsecskip{20pt} % Задаем отступ перед заголовком подсекции
\setaftersubsecskip{20pt} % Ставим такой же отступ после заголовка подсекции
\setsubsecheadstyle{\raggedright\normalfont\bfseries}  % Меняем стиль написания для заголовка подсекции: выравнивание по правому краю без переносов, нормальный размер, полужирный

%%% Задаем параметры оглавления %%%

\addto\captionsrussian{\renewcommand\contentsname{Содержание}} % Меняем слово "Оглавление" на "Содержание"
\setrmarg{2.55em plus1fil} % Запрещаем переносы слов в оглавлении
%\setlength{\cftbeforechapterskip}{0pt} % Эта команда убирает интервал между заголовками глав - тут не надо, так красивее смотрится
\renewcommand{\aftertoctitle}{\afterchaptertitle \vspace{-\cftbeforechapterskip}} % Делаем отступ между словом "Содержание" и первой строкой таким же, как у заголовков глав
%\renewcommand*{\chapternumberline}[1]{} % Делаем так, чтобы номер главы не печатался - тут не надо
\renewcommand*{\cftchapternumwidth}{1.5em} % Ставим подходящий по размеру разделитель между номером главы и самим заголовком
\renewcommand*{\cftchapterfont}{\normalfont\MakeUppercase} % Названия глав обычным шрифтом заглавными буквами
\renewcommand*{\cftchapterpagefont}{\normalfont} % Номера страниц обычным шрифтом
\renewcommand*{\cftchapterdotsep}{\cftdotsep} % Делаем точки до номера страницы после названий глав
\renewcommand*{\cftdotsep}{1} % Задаем расстояние между точками
\renewcommand*{\cftchapterleader}{\cftdotfill{\cftchapterdotsep}} % Делаем точки стандартной формы (по умолчанию они "жирные")
\maxtocdepth{subsection} % В оглавление попадают только разделы первыхтрех уровней: главы, секции и подсекции

%%% Выравнивание и переносы %%%

%% http://tex.stackexchange.com/questions/241343/what-is-the-meaning-of-fussy-sloppy-emergencystretch-tolerance-hbadness
%% http://www.latex-community.org/forum/viewtopic.php?p=70342#p70342
\tolerance 1414
\hbadness 1414
\emergencystretch 1.5em                             % В случае проблем регулировать в первую очередь
\hfuzz 0.3pt
\vfuzz \hfuzz
%\dbottom
%\sloppy                                            % Избавляемся от переполнений
\clubpenalty=10000                                  % Запрещаем разрыв страницы после первой строки абзаца
\widowpenalty=10000                                 % Запрещаем разрыв страницы после последней строки абзаца
\brokenpenalty=4991                                 % Ограничение на разрыв страницы, если строка заканчивается переносом

%%% Объясняем компилятору, какие буквы русского алфавита можно использовать в перечислениях (подрисунках и нумерованных списках) %%%
%%% По ГОСТ нельзя использовать буквы ё, з, й, о, ч, ь, ы, ъ %%%
%%% Здесь также переопределены заглавные буквы, хотя в принципе они в документе не используются %%%

\makeatletter
    \def\russian@Alph#1{\ifcase#1\or
       А\or Б\or В\or Г\or Д\or Е\or Ж\or
       И\or К\or Л\or М\or Н\or
       П\or Р\or С\or Т\or У\or Ф\or Х\or
       Ц\or Ш\or Щ\or Э\or Ю\or Я\else\xpg@ill@value{#1}{russian@Alph}\fi}
    \def\russian@alph#1{\ifcase#1\or
       а\or б\or в\or г\or д\or е\or ж\or
       и\or к\or л\or м\or н\or
       п\or р\or с\or т\or у\or ф\or х\or
       ц\or ш\or щ\or э\or ю\or я\else\xpg@ill@value{#1}{russian@alph}\fi}
\makeatother

% убираем бесячие варнинги из-за неиспоьзованных структур #myadd
\usepackage{silence}
\WarningsOff

%%% Задаем параметры оформления рисунков и таблиц %%%


\usepackage{graphicx, caption, subcaption} % Подгружаем пакеты для работы с графикой и настройки подписей
\graphicspath{{images/}} % Определяем папку с рисунками
\captionsetup[figure]{font=normalsize, width=\textwidth, name=Рисунок, justification=centering} % Задаем параметры подписей к рисункам: маленький шрифт (в данном случае 12pt), ширина равна ширине текста, полнотекстовая надпись "Рисунок", выравнивание по центру
\captionsetup[subfigure]{font=normalsize} % Индексы подрисунков а), б) и так далее тоже шрифтом 12pt (по умолчанию делает еще меньше)
\captionsetup[table]{singlelinecheck=false,font=normalsize,width=\textwidth,justification=justified} % Задаем параметры подписей к таблицам: запрещаем переносы, маленький шрифт (в данном случае 12pt), ширина равна ширине текста, выравнивание по ширине
\captiondelim{ --- } % Разделителем между номером рисунка/таблицы и текстом в подписи является длинное тире
\setkeys{Gin}{width=\textwidth} % По умолчанию размер всех добавляемых рисунков будет подгоняться под ширину текста
\renewcommand{\thesubfigure}{\asbuk{subfigure}} % Нумерация подрисунков строчными буквами кириллицы
%\setlength{\abovecaptionskip}{0pt} % Отбивка над подписью - тут не меняем
%\setlength{\belowcaptionskip}{0pt} % Отбивка под подписью - тут не меняем
\usepackage[section]{placeins} % Объекты типа float (рисунки/таблицы) не вылезают за границы секциии, в которой они объявлены

%%% Задаем параметры ссылок и гиперссылок %%% 

\usepackage{hyperref}                               % Подгружаем нужный пакет                       % Подгружаем нужный пакет
\hypersetup{
    colorlinks=true,                                % Все ссылки и гиперссылки цветные
    linktoc=all,                                    % В оглавлении ссылки подключатся для всех отображаемых уровней
    % linktocpage=true,                               % Ссылка - только номер страницы, а не весь заголовок (так выглядит аккуратнее)
    linkcolor=black,                                  % Цвет ссылок и гиперссылок - красный
    citecolor=red                                   % Цвет цитировний - красный
}

%%% Настраиваем отображение списков %%%

\usepackage{enumitem}                               % Подгружаем пакет для гибкой настройки списков
\renewcommand*{\labelitemi}{\normalfont{--}}        % В ненумерованных списках для пунктов используем короткое тире
\makeatletter
    \AddEnumerateCounter{\asbuk}{\russian@alph}     % Объясняем пакету enumitem, как использовать asbuk
\makeatother
\renewcommand{\labelenumii}{\asbuk{enumii})}        % Кириллица для второго уровня нумерации
\renewcommand{\labelenumiii}{\arabic{enumiii})}     % Арабские цифры для третьего уровня нумерации
\setlist{noitemsep, leftmargin=*}                   % Убираем интервалы между пунками одного уровня в списке
\setlist[1]{labelindent=\parindent}                 % Отступ у пунктов списка равен абзацному отступу
\setlist[2]{leftmargin=\parindent}                  % Плюс еще один такой же отступ для следующего уровня
\setlist[3]{leftmargin=\parindent}                  % И еще один для третьего уровня

%%% Счетчики для нумерации объектов %%%

\counterwithout{figure}{chapter}                    % Сквозная нумерация рисунков по документу
\counterwithout{equation}{chapter}                  % Сквозная нумерация математических выражений по документу
\counterwithout{table}{chapter}                     % Сквозная нумерация таблиц по документу

%%% Реализация библиографии пакетами biblatex и biblatex-gost с использованием движка biber %%%

\usepackage{csquotes} % Пакет для оформления сложных блоков цитирования (biblatex рекомендует его подключать)
\usepackage[%
backend=biber,                                      % Движок
bibencoding=utf8,                                   % Кодировка bib-файла
sorting=none,                                       % Настройка сортировки списка литературы
style=gost-numeric,                                 % Стиль цитирования и библиографии по ГОСТ
language=auto,                                      % Язык для каждой библиографической записи задается отдельно
autolang=other,                                     % Поддержка многоязычной библиографии
sortcites=true,                                     % Если в квадратных скобках несколько ссылок, то отображаться будут отсортированно
movenames=false,                                    % Не перемещать имена, они всегда в начале библиографической записи
maxnames=5,                                         % Максимальное отображаемое число авторов
minnames=3,                                         % До скольки сокращать число авторов, если их больше максимума
doi=false,                                          % Не отображать ссылки на DOI
isbn=false,                                         % Не показывать ISBN, ISSN, ISRN
]{biblatex}[2016/09/17]
\DeclareDelimFormat{bibinitdelim}{}                 % Убираем пробел между инициалами (Иванов И.И. вместо Иванов И. И.)
\addbibresource{biba.bib}                           % Определяем файл с библиографией

%%% Скрипт, который автоматически подбирает язык (и, следовательно, формат) для каждой библиографической записи %%%
%%% Если в названии работы есть кириллица - меняем значение поля langid на russian %%%
%%% Все оставшиеся пустые места в поле langid заменяем на english %%%

\DeclareSourcemap{
  \maps[datatype=bibtex]{
    \map{
        \step[fieldsource=title, match=\regexp{^\P{Cyrillic}*\p{Cyrillic}.*}, final]
        \step[fieldset=langid, fieldvalue={russian}]
    }
    \map{
        \step[fieldset=langid, fieldvalue={english}]
    }
  }
}

%%% Прочие пакеты для расширения функционала %%%

\usepackage{longtable,ltcaption}                    % Длинные таблицы
\usepackage{multirow,makecell}                      % Улучшенное форматирование таблиц
\usepackage{booktabs}                               % Еще один пакет для красивых таблиц
\usepackage{soulutf8}                               % Поддержка переносоустойчивых подчёркиваний и зачёркиваний
\usepackage{icomma}                                 % Запятая в десятичных дробях
\usepackage{hyphenat}                               % Для красивых переносов
\usepackage{textcomp}                               % Поддержка "сложных" печатных символов типа значков иены, копирайта и т.д.
\usepackage[version=4]{mhchem}                      % Красивые химические уравнения
\usepackage{amsmath}                                % Усовершенствование отображения математических выражений 

%%% Вставляем по очереди все содержательные части документа %%%

%#myadd
\usepackage{minted}
\setminted{breaklines,breakanywhere}
\usepackage{listings}
\usepackage{tabularx}
\usepackage{nicematrix}
\usepackage{graphicx}
\usepackage[absolute]{textpos}
\usepackage{amsfonts}
\usepackage{xargs}
\newcommandx{\img}[3][1=,2=,3=1]{
    \begin{figure}[H]
    \centering
    \includegraphics[width=#3\textwidth]{images/#1}
    \caption{#2}
    \end{figure}
}
\begin{document}

\thispagestyle{empty}
\begin{textblock*}{5cm}(15.2cm,13.4cm) % {ширина блока} (левый угол: X,Y)
    \includegraphics[width=3cm]{C:/Users/Den1ve/Pictures/документы/auth/auth.png} 
\end{textblock*}
\begin{center}
    \begin{sloppypar}
        {\fontsize{12}{14}\selectfont
        \textbf{{Министерство науки и высшего образования Российской Федерации}}\\
        \vspace{10pt}
        \textbf{ФЕДЕРАЛЬНОЕ ГОСУДАРСТВЕННОЕ АВТОНОМНОЕ ОБРАЗОВАТЕЛЬНОЕ\\УЧРЕЖДЕНИЕ ВЫСШЕГО ОБРАЗОВАНИЯ}\\
        «\textbf{НАЦИОНАЛЬНЫЙ ИССЛЕДОВАТЕЛЬСКИЙ УНИВЕРСИТЕТ ИТМО}»}
    \end{sloppypar}
    \vspace{30pt}

    \textbf{Факультет безопасности информационных технологий}
\end{center}

\vfill

\begin{center}
    \textbf{Дисциплина:} \\  
    \textit{"<Схемотехника">}

    \vspace{40pt}

    
    \uppercase{\textbf{Отчет по лабораторной работе № 1}}\\
    \textit{"<Исследование переходных процессов в линейных схемах">}

\end{center}

\vfill
 \begin{flushright}
    \hfill \textbf{Выполнили:}\\
    \vspace{15pt}
    \noindent Сыралёв И. А., студент группы N3346 \hfill $\underset{\text{(подпись)}}{\underline{\underline{\hspace{5cm}}}}$\\
    \vspace{15pt}
    \noindent Степанов Е. К., студент группы N3349 \hfill $\underset{\text{(подпись)}}{\underline{\underline{\hspace{5cm}}}}$\\
    \vspace{15pt}
    \noindent Шарапов А. В., студент группы N3351 \hfill $\underset{\text{(подпись)}}{\underline{\underline{\hspace{5cm}}}}$\\
    \vspace{15pt}
    \noindent Шевченко Н. А., студент группы N3348 \hfill $\underset{\text{(подпись)}}{\underline{\underline{\hspace{5cm}}}}$\\
    \vspace{15pt}
    \textbf{Проверил}: \\
    \vspace{10pt}
    \hfill {Гришенцев Алексей Юрьевич}\\
    \vspace{10pt}
    $\underset{\text{(отметка о выполнении)}}{\underline{\underline{\hspace{5cm}}}}$\\
    \vspace{22pt}
    $\underset{\text{(подпись)}}{\underline{\hspace{5cm}}}$
 \end{flushright}
\vfill
\begin{center}
    Санкт-Петербург \\
    2024
\end{center}                                     % Титульник

\newpage % Переходим на новую страницу
\setcounter{page}{2} % Начинаем считать номера страниц со второй
\OnehalfSpacing* % Задаем полуторный интервал текста (в титульнике одинарный, поэтому команда стоит после него)

\tableofcontents*                                   % Автособираемое оглавление

\chapter*{Введение}
\addcontentsline{toc}{chapter}{Введение}
\label{ch:intro}
Цель — исследовать переходные процессы в линейных электрических цепях.

Для достижения поставленной цели необходимо выполнить следующие задачи:
\begin{itemize}
    \item Правильно собрать лабораторную установку;
    \item Добиться устойчивого отображения одного периода переходного процесса на экране осциллографа;
    \item Зафиксировать полученные значения напряжений $\left(U_1, U_2, \dots \right)$, как минимум, в двух различных моментах времени $\left(t_1, t_2, \dots \right)$ на одном и том же участке кривой переходного процесса;
    \item Отключить питание приборов и разобрать схему лабораторной установки;
    \item Рассчитать по результатам измерений: значение постоянной времени, диапазон возможных значений сопротивления R в RC-цепи, если емкость конденсатора находится в диапазоне 1000 мкФ +60\%, -20\%;
    \item Рассчитать теоретический график переходного процесса для измеренного значения постоянной времени и сравнить с точками, не использовавшихся в его определении;
    \item Провести моделирование в Microcap.
\end{itemize}

Рабочие формулы:

Формула напряжения
\begin{equation}
\label{eq:e1}
U=E_0 \cdot e^{-\frac{t}{\tau}}
\end{equation}

Формула постоянного времени
\begin{equation}
\label{eq:e2}
\tau=RC
\end{equation}

Формула сопротивления
\begin{equation}
\label{eq:e3}
R = \frac{\tau}{C}
\end{equation}


\endinput                                     % Введение
\chapter{Ход работы}
\label{ch:chap1}

\section{Лабораторный стенд}

\img[1.png][Блок-схема лабораторной установки. 1) — источник питания, 2) — лабораторный стенд, 3) — осциллограф]

\img[2.png][Фотография лабораторного стенда]

\img[3.png][Схема лабораторной установки]

\section{Расчёты}

\begin{table}[H]
    \caption{Протокол измерений напряжения и времени}
\label{tab:t1}
\centering
 \begin{NiceTabular*}{\linewidth}{@{\extracolsep{\fill}}rrr}[hlines,vlines]
№ & $t$, $10^{-3}$с & $U$, В \\
\hline
1 & 0,56 & 3,32 \\
2 & 1,68 & 3,24 \\
3 & 4,84 & 3,12 \\
4 & 10 & 2,94 \\
5 & 15 & 2,64 \\
6 & 18 & 2,56 \\
7 & 21 & 2,4 \\
8 & 24 & 2,24 \\
9 & 27 & 2,2 \\
10 & 30 & 2,08 \\
11 & 33 & 1,96 \\
12 & 36 & 1,88 \\
13 & 39 & 1,8 \\
14 & 42 & 1,72 \\
15 & 45 & 1,64 \\
16 & 50 & 1,52 \\
17 & 56,4 & 1,4 \\
\end{NiceTabular*}
\end{table}

\img[4.png][График зависимости напряжения от времени]

Вычислим постоянное время и сопротивление по формулам \eqref{eq:e1} и \eqref{eq:e2}


Для уравнения \eqref{eq:e1} возьмём начальную и конечную точки с $U_1=3,32$ В, $U_2=1,4$ В и $t_1=0,56$ мс, $t_2=56,4$ мс соответственно.

\begin{equation*}
    \begin{cases}
        3.32 = U_0 \cdot e^{-\frac{0.56}{\tau}}\\
        1.4 = U_0 \cdot e^{-\frac{56.4}{\tau}}
    \end{cases}
    \begin{cases}
        \tau = 64.8 \text{ мс}\\
        U_0=3.35 \text{ В}
    \end{cases}
\end{equation*}

Используя формулу \eqref{eq:e3}, вычислим значения сопротивления.
\begin{table}[H]
    \caption{Зависимость сопротивления от ёмкости}
\label{tab:t1}
\centering
 \begin{NiceTabular*}{\linewidth}{@{\extracolsep{\fill}}rrr}[hlines,vlines]
№ & $C$, $10^{-6}$Ф & $R$, Ом \\
\hline
1 & 800 & 81 \\
2 & 1000 & 64,8 \\
3 & 1600 & 40,5 \\
\end{NiceTabular*}
\end{table}

Получаем диапазон сопротивлений $R \in \left[ 40,5; 81\right]$ Ом

Вычислим теоретическое напряжение для измеренного значения постоянной времени

\begin{table}[H]
    \caption{Расчёт теоретического напряжения}
\label{tab:t1}
\centering
 \begin{NiceTabular*}{\linewidth}{@{\extracolsep{\fill}}rrrr}[hlines,vlines]
№ & $t$, $10^{-3}$с & $U$, В & $U_{\text{теор}}$, В\\
\hline
1 & 0,56 & 3,32 & 3,26 \\
2 & 1,68 & 3,24 & 3,1 \\
3 & 4,84 & 3,12 & 2,87 \\
4 & 10 & 2,92 & 2,66 \\
5 & 15 & 2,64 & 2,53 \\
6 & 18 & 2,56 & 2,42 \\
7 & 21 & 2,4 & 2,31 \\
8 & 24 & 2,24 & 2,2 \\
9 & 27 & 2,2 & 2,1 \\
10 & 30 & 2,08 & 2 \\
11 & 33 & 1,96 & 1,92 \\
12 & 36 & 1,88 & 1,83 \\
13 & 39 & 1,8 & 1,75 \\
14 & 42 & 1,72 & 1,67 \\
15 & 45 & 1,64 & 1,55 \\
16 & 50 & 1,52 & 1,4 \\
17 & 56,4 & 1,4 & 1,4\\
\end{NiceTabular*}
\end{table}

\img[5.png][Графики фактической и теоретической зависимости напряжения от времени]
\section{Моделирование в MicroCap}
Значения в модели:\\
$C= 1000$ мкФ\\
$R= 64,8$ Ом\\
$U= 3,32$ В\\

\img[microcap.png][Схема лабораторного стенда, созданная в MicroCap]
\img[6.png][Анализ переходного процесса в microcap]
Для поиска значений производных поставим курсоры близко друг к другу
\img[7.png][Поиск значений производных]
Разница между курсорами: 2 мкс и -75мкВ


Найдём график производной и построим в тех же осях:
\img[8.png][График производной]
Разница во времени, равная постоянной времени переходного процесса, составляет 74.9 мс, что примерно равно результату, полученному практически.

\endinput                                     % Первая глава                                    % Вторая глава
\chapter*{Заключение}
\addcontentsline{toc}{chapter}{Заключение}
\label{ch:concl}

В ходе лабораторной работы мы исследовали переходные процессы в цепи, и установили, что фактический график зависимости напряжения от времени очень близок к теоретическому.

Были выполнены следующие задачи:
\begin{itemize}
    \item Лабораторная установка собрана;
    \item Получено устойчивое отображения одного периода переходного процесса на экране осциллографа;
    \item Зафиксированы значения напряжения в 17 точках на одном и том же участке кривой переходного процесса;
    \item По результатам измерений рассчитаны: значение постоянной времени, диапазон возможных значений сопротивления R в RC-цепи, если емкость конденсатора находится в диапазоне 1000 мкФ +60\%, -20\%;
    \item Рассчитан и построен теоретический график переходного процесса для измеренного значения постоянной времени;
    \item Провели моделирование в Microcap.

\end{itemize}
Все задачи выполнены успешно.

\endinput                                     % Третья глава
\nocite{*}                                          %позволяет выводть список литературы без ссылок в тексте
% \printbibliography[title=Список использованных источников] % Автособираемый список литературы

\end{document}