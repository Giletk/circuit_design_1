\chapter{Ход работы}
\label{ch:chap1}

\section{Лабораторный стенд}

\img[1.png][Блок-схема лабораторной установки]

\img[2.png][Фотография лабораторного стенда]

\img[3.png][Схема лабораторной установки]

\section{Расчёты}

\begin{table}[H]
    \caption{Протокол измерений напряжения и времени}
\label{tab:t1}
\centering
 \begin{NiceTabular*}{\linewidth}{@{\extracolsep{\fill}}rrr}[hlines,vlines]
№ & $t$, $10^{-3}$с & $U$, В \\
\hline
1 & 0,56 & 3,32 \\
2 & 1,68 & 3,24 \\
3 & 4,84 & 3,12 \\
4 & 10 & 2,94 \\
5 & 15 & 2,64 \\
6 & 18 & 2,56 \\
7 & 21 & 2,4 \\
8 & 24 & 2,24 \\
9 & 27 & 2,2 \\
10 & 30 & 2,08 \\
11 & 33 & 1,96 \\
12 & 36 & 1,88 \\
13 & 39 & 1,8 \\
14 & 42 & 1,72 \\
15 & 45 & 1,64 \\
16 & 50 & 1,52 \\
17 & 56,4 & 1,4 \\
\end{NiceTabular*}
\end{table}

\img[4.png][График зависимости напряжения от времени]

Вычислим постоянное время и сопротивление по формулам \eqref{eq:e1} и \eqref{eq:e2}


Для уравнения \eqref{eq:e1} возьмём начальную и конечную точки с $U_1=3,32$ В, $U_2=1,4$ В и $t_1=0,56$ мс, $t_2=56,4$ мс соответственно.

\begin{equation*}
    \begin{cases}
        3.32 = E_0 \cdot e^{-\frac{0.56}{\tau}}\\
        1.4 = E_0 \cdot e^{-\frac{56.4}{\tau}}
    \end{cases}
    \begin{cases}
        \tau = 64.8 \text{ мс}\\
        E_0=3.35 \text{ В}
    \end{cases}
\end{equation*}

Используя формулу \eqref{eq:e3}, вычислим значения сопротивления.
\begin{table}[H]
    \caption{Зависимость сопротивления от ёмкости}
\label{tab:t1}
\centering
 \begin{NiceTabular*}{\linewidth}{@{\extracolsep{\fill}}rrr}[hlines,vlines]
№ & $C$, $10^{-6}$Ф & $R$, Ом \\
\hline
1 & 800 & 81 \\
2 & 1000 & 64,8 \\
3 & 1600 & 40,5 \\
\end{NiceTabular*}
\end{table}

Получаем диапазон сопротивлений $R \in \left[ 40,5; 81\right]$ Ом

Вычислим теоретическое напряжение для измеренного значения постоянной времени

\begin{table}[H]
    \caption{Расчёт теоретического напряжения}
\label{tab:t1}
\centering
 \begin{NiceTabular*}{\linewidth}{@{\extracolsep{\fill}}rrrr}[hlines,vlines]
№ & $t$, $10^{-3}$с & $U$, В & $U_{\text{теор}}$, В\\
\hline
1 & 0,56 & 3,32 & 3,26 \\
2 & 1,68 & 3,24 & 3,1 \\
3 & 4,84 & 3,12 & 2,87 \\
4 & 10 & 2,92 & 2,66 \\
5 & 15 & 2,64 & 2,53 \\
6 & 18 & 2,56 & 2,42 \\
7 & 21 & 2,4 & 2,31 \\
8 & 24 & 2,24 & 2,2 \\
9 & 27 & 2,2 & 2,1 \\
10 & 30 & 2,08 & 2 \\
11 & 33 & 1,96 & 1,92 \\
12 & 36 & 1,88 & 1,83 \\
13 & 39 & 1,8 & 1,75 \\
14 & 42 & 1,72 & 1,67 \\
15 & 45 & 1,64 & 1,55 \\
16 & 50 & 1,52 & 1,4 \\
17 & 56,4 & 1,4 & 1,4\\
\end{NiceTabular*}
\end{table}

\img[5.png][Графики фактической и теоретической зависимости напряжения от времени]
\section{Моделирование в MicroCap}
Значения в модели:\\
$C= 1000$ мкФ\\
$R= 64,8$ Ом\\
$U= 3,32$ В\\

\img[microcap.png][Схема лабораторного стенда, созданная в MicroCap]
\img[6.png][Анализ переходного процесса в microcap]
Для поиска значений производных поставим курсоры близко друг к другу
\img[7.png][Поиск значений производных]
Разница между курсорами: 2 мкс и -75мкВ


Найдём график производной и построим в тех же осях:
\img[8.png][График производной]
Разница во времени, равная постоянной времени переходного процесса, составляет 74.9 мс, что примерно равно результату, полученному практически.

\endinput