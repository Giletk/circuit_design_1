\chapter*{Заключение}
\addcontentsline{toc}{chapter}{Заключение}
\label{ch:concl}

В ходе лабораторной работы мы исследовали переходные процессы в цепи, и установили, что фактический график зависимости напряжения от времени очень близок к теоретическому.

Были выполнены следующие задачи:
\begin{itemize}
    \item Лабораторная установка собрана;
    \item Получено устойчивое отображения одного периода переходного процесса на экране осциллографа;
    \item Зафиксированы значения напряжения в 17 точках на одном и том же участке кривой переходного процесса;
    \item По результатам измерений рассчитаны: значение постоянной времени, диапазон возможных значений сопротивления R в RC-цепи, если емкость конденсатора находится в диапазоне 1000 мкФ +60\%, -20\%;
    \item Рассчитан и построен теоретический график переходного процесса для измеренного значения постоянной времени;
    \item Провели моделирование в Microcap.

\end{itemize}
Все задачи выполнены успешно.

\endinput