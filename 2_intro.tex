\chapter*{Введение}
\addcontentsline{toc}{chapter}{Введение}
\label{ch:intro}
Цель — исследовать переходные процессы в линейных электрических цепях.

Для достижения поставленной цели необходимо выполнить следующие задачи:
\begin{itemize}
    \item Правильно собрать лабораторную установку;
    \item Добиться устойчивого отображения одного периода переходного процесса на экране осциллографа;
    \item Зафиксировать полученные значения напряжений $\left(U_1, U_2, \dots \right)$, как минимум, в двух различных моментах времени $\left(t_1, t_2, \dots \right)$ на одном и том же участке кривой переходного процесса;
    \item Отключить питание приборов и разобрать схему лабораторной установки;
    \item Рассчитать по результатам измерений: значение постоянной времени, диапазон возможных значений сопротивления R в RC-цепи, если емкость конденсатора находится в диапазоне 1000 мкФ +60\%, -20\%;
    \item Рассчитать теоретический график переходного процесса для измеренного значения постоянной времени и сравнить с точками, не использовавшихся в его определении;
    \item Провести моделирование в Microcap.
\end{itemize}

Рабочие формулы:

Формула напряжения
\begin{equation}
\label{eq:e1}
U=E_0 \cdot e^{-\frac{t}{\tau}}
\end{equation}

Формула постоянного времени
\begin{equation}
\label{eq:e2}
\tau=RC
\end{equation}

Формула сопротивления
\begin{equation}
\label{eq:e3}
R = \frac{\tau}{C}
\end{equation}


\endinput